\documentclass{purdue-poster}

% Optionally change the font size to fit more/less text on the poster.
\usepackage[fontsize=40pt]{fontsize}
% Paper size
\geometry{papersize={40in, 32in}}

% For filler text:
\usepackage[base]{babel}
\usepackage{lipsum}

% Optionally change the title size to fit the length
\title{\Huge{A Purdue \LaTeX\ Poster Template}}
\author{\Large{A Purdue Student\texorpdfstring{\textsuperscript{1}}{}, A Duepur Student\texorpdfstring{\textsuperscript{2}}{}}}
\institute
{\large{Purdue University\texorpdfstring{\textsuperscript{1}}{}, Duepur University\texorpdfstring{\textsuperscript{2}},\\
Appearing at Overleaf Template}}
\date{\today}

\renewcommand{\titlelogo}{
    % You can put multiple logos here.
    \includesvg[width=.06\paperwidth,keepaspectratio]{logo/cs-v-rev.svg}\par 
    \vspace{4ex}
    \includesvg[width=.06\paperwidth,keepaspectratio]{logo/pu-v-rev.svg}\par
}

% Optionally set some colsep
\newlength{\postercolsep}
\setlength{\postercolsep}{1em}
\newcommand{\postercolwidth}[1]{\dimeval{#1-\postercolsep}}

\begin{document}
\begin{frame}{}
\begin{columns}[onlytextwidth, t]
    % Feel free to twist the line widths for different layout.
    \begin{column}{\postercolwidth{.6\linewidth}}

        \begin{abstractblock}{Overview}
            Nobody actually reads a very long paragraph in a poster.

            \bigskip

            \lipsum[2]
        \end{abstractblock}

        \begin{columns}[onlytextwidth, t]
        
            \begin{column}{\postercolwidth{.5\linewidth}}
                
                \begin{plainblock}{Itemize List}
                    Some introduction of the list.
                    \begin{itemize}
                        \item Bulleted copy. Keep it short with bite-size chunks of information.
                        \begin{itemize}
                            \item Bulleted copy. Keep it short with bite-size chunks of information.
                        \end{itemize}
                        \item Bulleted copy. Keep it short with bite-size chunks of information.
                        \item Bulleted copy. Keep it short with bite-size chunks of information.
                    \end{itemize}
                \end{plainblock}
            
                \begin{plainblock}{Enumerate List}
                    Some introduction of the list.
                    \begin{enumerate}
                        \item Bulleted copy.
                        \begin{enumerate}
                            \item Bulleted copy.
                        \end{enumerate}
                        \item Bulleted copy.
                    \end{enumerate}
                \end{plainblock}

            \end{column}

            \begin{column}{\postercolwidth{.5\linewidth}}
                
                \begin{plainblock}{Maths. It's fine if the heading takes two lines}
                    An example of some very long equations with $\Psi (x,t)$:
                    
                    \begin{align}
                    i\hbar {\frac {\partial }{\partial t}}\Psi (x,t)&=\left[-{\frac {\hbar ^{2}}{2m}}{\frac {\partial ^{2}}{\partial x^{2}}}+V(x,t)\right]\Psi (x,t) \\
                    i\hbar {\frac {d}{dt}}\vert \Psi (t)\rangle &={\hat {H}}\vert \Psi (t)\rangle \\ 
                    |\Psi (t)\rangle &=\sum _{n}A_{n}e^{{-iE_{n}t}/\hbar }|\psi _{E_{n}}\rangle
                    \end{align}
                    
                    Indeed an example of some very long equations with $\Psi (x,t)$.
                \end{plainblock}

                \vspace{1em}

                \begin{block}{Do You Know}
                    This is a gold block with two different colors.
                    
                    Good use of colors makes the poster less monotone.
                \end{block}

            \end{column}
            
        \end{columns}
    
    \end{column}

    \begin{column}{\postercolwidth{.4\linewidth}}
        
        \begin{plainblock}{Figure}
            \LaTeX\ can draw figures with the tikz package:

            {
                \centering
                \small
                \begin{figure}[h]
                    \centering
                    \begin{tikzpicture}[
                    square/.style={rectangle, draw=black!, very thick, minimum size=5mm},
                    ]
                    \node[square](client){\footnotesize Client};
                    \node[square](r1)[right=2em of client]{\footnotesize Relay 1};
                    \node[square](r2)[right=2em of r1]{\footnotesize Relay 2};
                    \node[square](r3)[right=2em of r2]{\footnotesize Relay 3};
                    \node[square](server)[right=2em of r3]{\footnotesize Server};
                    \draw[->](client.east)--node[above=1em]{\footnotesize$E_1(E_2(E_3(P)))$}(r1.west);
                    \draw[->](r1.east)--node[above=1em]{\footnotesize$E_2(E_3(P))$}(r2.west);
                    \draw[->](r2.east)--node[above=1em]{\footnotesize$E_3(P)$}(r3.west);
                    \draw[->](r3.east)--node[above=1em]{\footnotesize$P$}(server);
                    \end{tikzpicture}
                    \caption{An Example of a Three-Hop Connection}
                    \label{fig:three-hop}
                \end{figure}
            }

        \end{plainblock}

        \begin{exampleblock}{Block with Another Color}
        A gray block with two different colors.
        \end{exampleblock}

        \begin{alertblock}{Alert Block}
        A red block for alert, in default theme from \LaTeX.
        \end{alertblock}

    \begin{plainblock}{Thank you for using!}
        For issues on the template, please visit the Github page:
        
        {\small\texttt{https://github.com/zhtluo/purdue-slide-template}\par}
    \end{plainblock}

    \begin{plainblock}{QR Code}
        Put a QR code here to give the reader something to scan.

        You can find the SVG QR generator under \texttt{qr.py}.

        \begin{figure}
            {\centering\includesvg[width=8em]{qrcode.svg}}
        \end{figure}
    \end{plainblock}
    \end{column}
    
\end{columns}
\end{frame}
\end{document}
