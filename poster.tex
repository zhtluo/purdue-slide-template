\documentclass{purdue-poster}

% Paper size
\usepackage[fontsize=38pt]{fontsize}
\geometry{papersize={32in, 40in}}

% For filler text:
\usepackage[base]{babel}
\usepackage{lipsum}

\title{\Huge{A Purdue \LaTeX\ Poster Template}}
\author{\Large{A Purdue Student\texorpdfstring{\textsuperscript{1}}{}, A Duepur Student\texorpdfstring{\textsuperscript{2}}{}}}
\institute
{\large{Purdue University\texorpdfstring{\textsuperscript{1}}{}, Duepur University\texorpdfstring{\textsuperscript{2}},\\
Appearing at Overleaf Template}}
\date{\today}

\renewcommand{\titlelogo}{
    % You can put multiple logos here.
    \includesvg[width=.06\paperwidth,keepaspectratio]{logo/cs-v-rev.svg}
    \vskip50pt
    \includesvg[width=.06\paperwidth,keepaspectratio]{logo/pu-v-rev.svg}
}

\begin{document}
\begin{frame}{}
    \begin{columns}[c]
    \begin{column}{.48\linewidth}
    \begin{block}{\large Long Text}
        Nobody actually reads a very long paragraph in a poster.

        \bigskip

        \lipsum[2]
    \end{block}

    \begin{block}{Itemize List}
        Some introduction of the list.
        \begin{itemize}
            \item Bulleted copy. Keep it short with bite-size chunks of information.
            \begin{itemize}
                \item Bulleted copy. Keep it short with bite-size chunks of information.
            \end{itemize}
            \item Bulleted copy. Keep it short with bite-size chunks of information.
            \item Bulleted copy. Keep it short with bite-size chunks of information.
        \end{itemize}
    \end{block}

    \begin{block}{Enumerate List}
        Some introduction of the list.
        \begin{enumerate}
            \item Bulleted copy. Keep it short with bite-size chunks of information.
            \begin{enumerate}
                \item Bulleted copy. Keep it short with bite-size chunks of information.
            \end{enumerate}
            \item Bulleted copy. Keep it short with bite-size chunks of information.
            \item Bulleted copy. Keep it short with bite-size chunks of information.
        \end{enumerate}
    \end{block}

    \begin{block}{Maths}
        An example of some very long equations with $\Psi (x,t)$:
        
        \begin{align}
        i\hbar {\frac {\partial }{\partial t}}\Psi (x,t)&=\left[-{\frac {\hbar ^{2}}{2m}}{\frac {\partial ^{2}}{\partial x^{2}}}+V(x,t)\right]\Psi (x,t) \\
        i\hbar {\frac {d}{dt}}\vert \Psi (t)\rangle &={\hat {H}}\vert \Psi (t)\rangle \\ 
        |\Psi (t)\rangle &=\sum _{n}A_{n}e^{{-iE_{n}t}/\hbar }|\psi _{E_{n}}\rangle
        \end{align}
        
        Indeed an example of some very long equations with $\Psi (x,t)$.
    \end{block}
    
    \end{column}

    \begin{column}{.48\linewidth}
    
    \begin{block}{Figure}
        \LaTeX\ can draw figures with the tikz package:
        \begin{figure}[h]
        \centering
        \begin{tikzpicture}[
        square/.style={rectangle, draw=black!, very thick, minimum size=5mm},
        ]
        \node[square](client){\footnotesize Client};
        \node[square](r1)[right=2em of client]{\footnotesize Relay 1};
        \node[square](r2)[right=2em of r1]{\footnotesize Relay 2};
        \node[square](r3)[right=2em of r2]{\footnotesize Relay 3};
        \node[square](server)[right=2em of r3]{\footnotesize Server};
        \draw[->](client.east)--node[above=1em]{\footnotesize$E_1(E_2(E_3(P)))$}(r1.west);
        \draw[->](r1.east)--node[above=1em]{\footnotesize$E_2(E_3(P))$}(r2.west);
        \draw[->](r2.east)--node[above=1em]{\footnotesize$E_3(P)$}(r3.west);
        \draw[->](r3.east)--node[above=1em]{\footnotesize$P$}(server);
        \end{tikzpicture}
        \caption{An Example of a Three-Hop Connection}
        \label{fig:three-hop}
        \end{figure}
    \end{block}

    \begin{exampleblock}{Block with Another Color}
    A gray block with two different colors.
    \end{exampleblock}

    \begin{alertblock}{Alert Block}
    A red block for alert, in default theme from \LaTeX.

    \bigskip
    
    \lipsum[7]
    \end{alertblock}

    \begin{block}{Thank you for using!}
        For issues on the template, please visit the Github page:
        
        {\small\texttt{https://github.com/zhtluo/purdue-slide-template}\par}
    \end{block}

    \begin{block}{\large QR Code}
        Put a QR code here to give the reader something to scan.

        You can find the SVG QR generator under \texttt{qr.py}.

        \begin{figure}
            {\centering\includesvg[width=.5\linewidth]{qrcode.svg}}
        \end{figure}
    \end{block}
    \end{column}
    \end{columns}
    \vfill
\end{frame}
\end{document}
